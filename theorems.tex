
\newtheoremstyle{style}
   {0.5cm}                   %Space above
   {0cm}                   %Space below
   {\normalfont}                      %Body font: original {\normalfont}
   {}                      %Indent amount \left(empty = no indent,
                           %\parindent = para indent\right)
   {\normalfont\bfseries}  %Thm head font original {\normalfont\bfseries}
   {:}                     %Punctuation after thm head original :
   {\newline}              %Space after thm head: " " = normal interword
                            %space; \newline = linebreak
   {{\textbf{\thmname{#1}\thmnumber{ #2}\thmnote{ \left(#3\right)}}}} %Hier wird die endgültige Struktur deiner Umgebung festgelegt. Hier funktioniert vspace.                    
                                        %Thm head spec \left(can be left empty, meaning
                           %`normal'\right) original {\underline{\thmname{#1}\thmnumber{ #2}\thmnote{ \left(#3\right)}}}

\theoremstyle{style}

\newtheorem{definition}{Definition}
\newtheorem{satz}{Satz}
\newtheorem{lemma}[satz]{Lemma} % Lemma numbering together with theorem
\newtheorem{korollar}[satz]{Korollar} % Corollary numbering together with theorem
\newtheorem{proposition}[satz]{Proposition} % Proposition numbering together with theorem
\newtheorem{bemerkung}{Bemerkung}
\newtheorem{bsp}{Beispiel}

\renewcommand{\thebemerkung}{\hspace{-0.3em}}
\renewcommand{\thebsp}{\hspace{-0.3em}}
\renewcommand{\thedefinition}{\hspace{-0.3em}}
