\newtheoremstyle{style}
   {0.5cm}                 %Space above
   {0cm}                   %Space below
   {\normalfont}                      %Body font: original {\normalfont}
   {}                      %Indent amount \left(empty = no indent,
                           %\parindent = para indent\right)
   {\normalfont\bfseries}  %Thm head font original {\normalfont\bfseries}
   {:}                     %Punctuation after thm head original :
   {\newline}              %Space after thm head: " " = normal interword
			   %space; \newline = linebreak
  {\textbf{\thmname{#1}\thmnumber{ #2}}\thmnote{ (#3)}}
%Hier wird die endgültige Struktur deiner Umgebung festgelegt. Hier funktioniert vspace.                    
                                        %Thm head spec \left(can be left empty, meaning
                           %`normal'\right) original {\underline{\thmname{#1}\thmnumber{ #2}\thmnote{ \left(#3\right)}}}

\theoremstyle{style}

\newtheorem*{bsp}{Beispiel}
\newtheorem*{definition}{Definition}
\newtheorem{satz}{Satz}[section]
\newtheorem{lemma}{Lemma}
\newtheorem{korollar}{Korollar}
\newtheorem{proposition}{Proposition}
\newtheorem*{bemerkung}{Bemerkung}
\newtheorem*{anwendung}{Anwendung}
\newtheorem*{folgerung}{Folgerung}
\newtheorem*{kin}{Kinematische Interpretation}

%neue Beweisumgebung
\newtheorem*{beweisl}{Beweis}
\newenvironment{beweis}{\begin{beweisl}}{\begin{flushright}
                                          \qedsymbol
                                         \end{flushright}\end{beweisl}}

\makeatletter
\def\Links{\tagsleft@true} \def\Rechts{\tagsleft@false} 
\makeatother
