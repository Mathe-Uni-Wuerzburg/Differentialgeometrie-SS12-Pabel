\chapter{Lokale Kurventheorie im euklidischen Raum}
\section{Grundbegriffe der Kurventheorie}
Wir betrachten zunächst (kurzzeitig) rein affingeometrische Begriffe/Invarianten.
\begin{definition}
 Ein \uline{\(\C^r\)-Weg} oder eine \uline{parametrisierte \(\C^r\)-Kurve} (\(r\ge 0\)) [\(\C^r = r\)-mal stetig differenzierbar] im (affinen) \(\IR^n\) ist eine \(\C^r\)-Abbildung 
\[
 c: t\in I \subset \IR \mapsto c(t)\in \IR^n
\]
eines offenen Intervalls \(I\) in den \(\IR^n\). \\
\(t\) heißt \uline{Parameter}, die Bildmenge \(c[I] \subset \IR^n\) die \uline{Spur des Weges}. \\
Ein \(\C^r\)-Weg (\(n\ge 1\)) heißt \uline{regulär}, wenn überall der \uline{Tangentenvektor} \(\dot c(t) = \frac{\dd c}{\dd t}(t) \ne 0\) ist. Nichtreguläre Punkte \(c(t_0)\) mit \(\dot c(t_0)=0\) heißen \uline{Singularitäten}.
\end{definition}
\uline{Kinematische Interpretation:} \\
\(t \mapsto c(t)\) beschreibt die \uline{zeit}abhängige Bewegung eines Punktes im \(\IR^n\).
\(\dot c\) ist die vektorielle Geschwindigkeit (und im euklidischen \(\IR^n\) \(w:= | \dot c|\) die skalare Geschwindigkeit).

\begin{bsp}\(\)
\begin{enumerate}
 \item \uline{Peano-Kurve}: Stetiger (\(\C^0\)-)Weg im \(\IR^2\), dessen Spur jeden Punkt eines Gebietes \(G\subseteq \IR^2\) ausfüllt (niergends differenzierbar, "`unbrauchbar"')
 \item \uline{Konstanter Weg}: \(t \in I \mapsto c(t)=x_0 \in \IR^n\) (niergends regulär, "`unbrauchbar"')
 \item \uline{Neil'sche Parabel}: \(c: t\in \IR \mapsto c(t) = \begin{pmatrix}
                                                                t^2 \\
								t^3
                                                               \end{pmatrix}
								\in \IR^2 \)\quad (\(\C^\infty\)-Weg), in \(c(0)=\begin{pmatrix}
								                                                  0 \\
														  0
								                                                 \end{pmatrix} \) nicht regulär ("`Spitze"') (\(w(0)=|\dot c (0)|=0\), "`man hat Zeit, sich umzudrehen"')
 \item \uline{Kreislinie}: \(c: t\in \IR \mapsto c(t)=\begin{pmatrix}
                                                       \cos t \\
						       \sin t
                                                      \end{pmatrix} \in \IR^2\) ( \(\infty\)-oft durchlaufbar) [Affin gesehen ist das eine Ellipse!] \\
Aber auch \(t \mapsto \tilde c(t) = \begin{pmatrix}
                                     t \\
				     \pm \sqrt{1-t^2}
                                    \end{pmatrix} \) und \(t \mapsto \tilde{\tilde c}(t)=\begin{pmatrix}
											  \frac{1}{\cosh t} \\
											  \tanh t
											 \end{pmatrix} \)
sind Parametrisierungen von Kreisstücken.
\end{enumerate}
\end{bsp}
Wege,  die nur mit veränderlicher "`Zeitskala"' durchlaufen werden, sollen nicht als verschieden angesehen werden.
\begin{definition}
 \(I, \tilde I \subset \IR\) seien offene Intervalle. \\
Zwei Wege \(c: I \to \IR^n, \tilde c: \tilde I \to \IR^n\) heißen \uline{\(C^r\)-äquivalent} (\(r\ge 0 \)), wenn ein orientierungstreuer (d.h. monoton wachsender) \(\C^r\)-Diffiomorphismus \(\Phi : I \to \tilde I\) existiert, mit
\[
 \uline{c= \tilde c \circ \Phi}, \text{ d.h. } \uline{\forall_t c(t)=\tilde c (\Phi(t))}
\]

\end{definition}

\begin{bemerkung} \(\)
 \begin{enumerate}
  \item[0.] \(\Phi \, \C^r\)-Diffeomorphismus \(\Leftrightarrow \Phi\) bijektiv und \(\Phi\) \uline{und \(\Phi^{-1}\)}\, \(C^r\)-differenzierbar. 
  [Bsp.: \(\Phi: t \in \IR \to t^3 \in \IR\) ist \uline{kein} \(\C^1\)-Diffeomorphismus] \\
  Bei \(C^r\)-Diffeomorphismus ist stets \(\dot \Phi(t)\ne 0\) (falls \(r\ge 1\))
  \item[1.] \(\Phi\) ist (für \(r\ge 1\)) genau dann orientierungstreu, wenn überall \(\dot \Phi(t)>0\) ist.
  \item[2.] Äquivalente Wege besitzen (für \(r\ge 1\)) das gleiche Regularitätsverhalten.
  \[
   \dot c(t)= \dot{\tilde c} (\Phi(t)) \cdot \underbrace{\dot \Phi(t)}_{>0}
  \]
  \item[3.] Die Äquivalenz von Wegen ist wirklich eine Äquivalenzrelation (reflexiv, symmetrisch, transitiv)
 \end{enumerate}
\end{bemerkung}

\begin{definition}
 Eine (orientierte, reguläre) \uline{\(\C^r\)-Kurve} (\(r\ge1\)) im (affinen) \(\IR^n\) ist eine Äquivalenzklasse \([c]\) von regulären \(\C^r\)-Wegen \(c : I \subset \IR \to \IR^n\). Ein Repräsentant heißt eine (zulässige) \uline{Parametrisierungen} der \(\C^r\)-Kurve, eine die Äquivalenz vermittelnde Abbildung \(\Phi\) eine (zulässige) \uline{Parametertransformation}.
\end{definition}

\begin{bsp}
 Die "`Kreis"'-Darstellungen \[t \mapsto c(t)=\begin{pmatrix}
                                              \cos t \\
                                              \sin t
                                             \end{pmatrix} \in \IR^2, \left(|t|<\frac\pi2\right)
			     \]
 und \[\tilde t\mapsto \tilde c(\tilde t) = \begin{pmatrix}
                                             \frac1{\cosh \tilde t} \\
                                             \tanh \tilde t
                                            \end{pmatrix} \in \IR^2 (\tilde t \in \IR) 
     \]
 sind \(\C^\infty\)-äquivalente Parametertransformationen: \[
                                                            \Phi(t) = \operatorname{Artanh} \sin t = \tilde t
                                                           \]
mit
\[
 \dot \Phi(t)=\frac{\cos t}{1-\sin^2 t}= \frac1{\cos t} >0
\]

\end{bsp}
