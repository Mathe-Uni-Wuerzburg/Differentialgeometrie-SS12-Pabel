\chapter{Grundbegriffe und Bezeichnungen aus der linearen Algebra und analytischen Geometrie}
  
  Die klassische Differentialgeometrie der Kurven und Flächen benutzt als umgebenden Raum einen \(n\)-dimensionalen, orientierten, euklidischen Raum \(E^n\) mit zugehörigem euklischem Richtungsvektorraum \(V^n\)\index{Vektor}.

\section{Strukturen}
\(V^n\) ist mit einem Skalarprodukt\index{Skalarprodukt} \((X,Y) \mapsto \langle X, Y \rangle \in \IR\) ausgestattet. 
Damit lassen sich messen:
  \begin{itemize}
   \item die Länge\index{Vektor!-länge} von Vektoren \(X\): \(|X| = \sqrt{ \langle X,X \rangle }\)
   \item die Orthogonalität\index{Vektor!orthogonal} von Vektoren \(X, Y\): \(X \perp Y \Leftrightarrow \langle X, Y \rangle = 0\)
   \item der Winkel\index{Vektor!Winkel} zwischen zwei Vektoren \(X, Y\): \(\cos \angle (X, Y) = \left\langle \frac{X}{|X|}, \frac{Y}{|Y|} \right\rangle \)
%    \item der Abstand\index{Vektor!Abstand} von Punkten \(p, q\): \(\dd(p,q) = | \vv{pq} | \)
   \item Flächeninhalte, Volumina, usw.
  \end{itemize}
Ist zusätzlich eine feste Orthonormalbasis\index{Orthonormalbasis} \( (\mathring{e_1}, ... \mathring{e_n}) \) (definiert durch \( \langle \mathring{e_i}, \mathring{e_k} \rangle = \delta_{ik} \)) ausgezeichnet als positiv orientiert, erhält man eine Orientierung des Raumes und kann alle Basen in positiv und negativ orientierte einteilen.
  
\paragraph*{Standard-Modell:} \(E^n = V^n = \IR^n\), ausgestattet mit dem Standard-Skalarprodukt\index{Standard-Skalarprodukt} \(\langle X, Y \rangle = \sum_{i=1}^n X^iY^i\) und der (positiv orientierten) Standardbasis\index{Standardbasis} \(\mathring{e_1}, ... \mathring{e_n})\) mit \(\mathring{e_i} = (0, \dots, 1, \dots, 0)^T\).
Dieses Standardmodell reicht bei uns meist aus:
Bezüglich eines kartesischen Koordinatensystems\index{kartesisches Koordinatensystem} \((0; e_1, ... e_n)\) in einem abstrakten, orientierten euklidschen Raum \(E^n\), bestehend aus 
  \begin{itemize}
   \item einem "`Ursprung"' ("`Nullpunkt"') \(0 \in E^n\)
   \item einer positiv orientierten Orthonormalbasis \((e_1, ... e_n)\) im \(V^n\)
  \end{itemize}
kann man jedem Punkt und jedem Vektor eindeutig reelle Koordinaten zuordnen:
  \begin{itemize}
   \item Vektor: \(X = \sum_{i=1}^n X^ie_i \in V^n \mapsto (X^1, ... X^n) \in \IR^n\)
   \item Punkt: \(p = 0 + \sum_{i=1}^n p^ie_i \mapsto (p^1, ... p^n) \in \IR^n\)
  \end{itemize}
Aus einem Skalarprodukt in \(V^n\) wird in Koordinaten
  \[
   \langle X, Y \rangle = \left\langle \sum X^ie_i, \sum Y^ke_k \right\rangle = \sum_{i}\sum_{k} X^iY^k \langle e_i, e_k \rangle = \sum_{i=1}^n X^iY^i
  \]
das Standard-Skalarprodukt im \(\IR^n\).
Man ist im Stanard-Modell angelangt.
Ein Wechsel des kartesischen Koordinatensystems im \(E^n\) induziert im Koordinatenraum \(\IR^n\) eine \defNotion{Bewegung} \[p \mapsto p' = Dp + t\] bestehend aus einer eigentlichen orthogonalen Drehmatrix \(D \in SO(u, \IR)\) mit \(\det D = +1\) und einem Translationsvektor \(t \in \IR^n\).
In der euklidschen Differentialgeometrie werden Eigenschaften von Objekten (Kurven, Flächen, ...) untersucht, die invariant gegenüber solchen Transformationen sind, also nicht vom gewählten kartesischen Koordinatensystem abhängig sind. \\
  \begin{bemerkung} In der sogenannten affinen\index{affin} Differentialgeometrie untersucht man Eigenschaften von Objekten, die (in Koordinaten) invariant sind gegenüber beliebigen affinen Transformationen \(p \mapsto p' = Ap + t\), \(A\) regulär. Man ignoriert dort vollständig die metrische Struktur des \(\IR^n\).
  Der umgebende Raum ist dann ein \defNotion{affiner Punktraum} (bei uns nur am Rande betrachtet).
  \end{bemerkung}
Zum \uline{Vektorprodukt\index{Vektorprodukt}} (Kreuzprodukt\index{Kreuzprodukt}) im orientierten euklidischen \(\IR^n\): \\
  Zu je \(n-1\) Vektoren \(X_1, \dots, X_{n-1} \in \IR^n (n\ge 2)\) gibt es genau einen Vektor \(Y\in \IR^n\) mit den Eigenschaften
  \begin{enumerate}
   \item \(Y \perp X_k, (k = 1, \dots, n-1) \)
   \item \(|Y|=a_{n-1}(X_1, \dots, X_{n-1}) = \sqrt{\det\left(\langle X_i, X_k \rangle\right)_{i=k=1,\dots,n-1}} \) \\
	  \(= (n-1)\)-dimensionaler Flächeninhalt des von \(X_1, \dots, X_{n-1}\) aufgespannten \(n-1\)-dimensionalen Parallelogramms \\
	  \(=\) Wurzel aus der \uline{Gramschen} Determinanten \(G(X_1,\dots, X_{n-1})\)
   \item \(\det (X_1, \dots, X_{n-1}, Y) \ge 0\) (d.h. \((X_1, \dots, X_{n-1}, Y)\) ist positiv orientiert)
  \end{enumerate}
Bezeichnung: \(Y=X_1 \times \dots \times X_{n-1}\) \\
Eine explizite Formel ist (mit der Standardbasis \((e_1, \dots , e_n)\) des \(\IR^n\) ): 
\begin{align*}
 X_1 \times \dots \times X_{n-1} &= \sum_{i=1}^n \det (X_1, \dots, X_{n-1},e_i)e_i \\
				 &= \sum_{i=1}^n \left\vert\begin{matrix}
                                                            X_1^1  &\cdots  &X_{n-1}^1 &0 &\\
                                                            \vdots &       & \vdots   &\vdots &\\
                                                            \vdots &       & \vdots   &1 &(i)\\
                                                            \vdots &       & \vdots   & \vdots &\\
                                                            X_1^n  & \cdots &X_{n-1}^n &0 &
                                                           \end{matrix}
                                                           \right\vert  e_i = \left\vert\begin{matrix}
										         X_1^1 & \cdots & X_{n-1}^1 & e_1 \\
                                                                                         \vdots & & \vdots & \vdots \\
                                                                                         \vdots & & \vdots & \vdots \\
                                                                                         \vdots & & \vdots & \vdots \\
                                                                                         X_1^n & \cdots & X_{n-1}^n & e_n
                                                                                        \end{matrix}
                                                                                        \right\vert
\end{align*}
\begin{bsp}{\(\uline{n=2}\)}
 \[ X=\begin{pmatrix}
       X^1 \\
       X^2
      \end{pmatrix}
      \Rightarrow X^x = \left\vert \begin{matrix}
                                    X^1 & e_1 \\
                                    X^2 & e_2
                                   \end{matrix}
			\right\vert
			= -X^2 e_1 + X^1 e_2 = \begin{pmatrix}
			                        -X^2 \\
			                        X^1
			                       \end{pmatrix}
\]
\[
 |X^x| = a_1(X) = |X|
\]
\end{bsp}

\begin{bsp}{\(\uline{n=3}\):}
  \begin{align*}
   X \times Y = \left\vert \begin{matrix}
                            X^1 & Y^1 & e_1 \\
                            X^2 & Y^2 & e_2 \\
                            X^3 & Y^3 & e_3
                           \end{matrix}
		\right\vert
		=(X^2Y^3-X^3Y^2) e_1 + \dots
  \end{align*}
\[
 |X \times Y| = a_2(X,Y) = \sqrt{\det \begin{pmatrix}
                                      \langle X,X\rangle & \langle X,Y\rangle \\
                                      \langle Y,X \rangle & \langle Y,Y\rangle
                                     \end{pmatrix}
}
\]
\end{bsp}

\begin{anwendung} 
Jedes Orthonormalsystem \((e_1, \dots, e_{n-1})\) im \(\IR^n\) lässt sich durch \(e_n := e_1 \times \dots \times e_{n-1}\) eindeutig zu einer positiv orientierten Orthonormalbasis \((e_1, \dots, e_n)\) ergänzen.
\end{anwendung}
