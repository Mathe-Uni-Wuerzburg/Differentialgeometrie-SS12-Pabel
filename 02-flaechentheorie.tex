  \chapter{Lokale Flächentheorie im euklidischen Raum}
\section{Grundbegriffe der Flächentheorie}
\setcounter{subsection}{-1}
\subsection{$p$-dimensionale Flächen im affinen $\IR^n$}
Sei \(1 < p < n\) fest gewählt (in Anwendungen meist \(p = 2, n= 3\)).
 
\begin{definition}[Parametrisierte \(\C^r\)-\(p\)-Fläche (\(r \ge 0\))]\index{Fläche}
\(\C^r\)-Abbildung \(x\colon G \subset \IR^p \to \IR^n\)
\[
 u = \left( u^1, \dots , u^p \right) \mapsto x(u) = \left(x^1(u), \dots, x^n (u) \right) 
\]
wobei \(G\) ein \uline{Gebiet} des \(\IR^p\) (d.h. offen und zusammenhängend) ist.\\
\uline{Parameter:}\index{Fläche!Parameter} \[ u^1, \dots , u^p \quad \text{bei }n = 3\text{ meist }(u,v) \]
\uline{Parameterlinien:}\index{Fläche!Parameterlinie}
\[u^\varrho \mapsto x \left(u^1_0, \dots , u^\varrho , \dots , u^p_0 \right)\]
\uline{Spur:}\index{Fläche!Spur} \[M:= x\left[G \right] \subset \IR^n\]
\uline{Regularität \(r \geq 1\)}\index{Fläche!regulär}: Die partiellen Ableitungen 
\[x_\varrho := \partial_\varrho x = \frac{\partial x}{\partial u^\varrho} \quad (\varrho = 1, \dots , p)\] 
sind überall linear unabhängig (sonst \uline{Singularitäten})\index{Fläche!Singularität}.
\end{definition}

\begin{bemerkung}\(\)
\begin{enumerate}
 \item Regularität bedeutet: Die partiellen Ableitungen \(x_\varrho (\varrho = 1, \ldots , p)\) [\underline{"`Tangentialvektoren"'}] spannen überall einen p-dimensionalen \uline{Tangentialraum}\index{Tangentialraum} auf. Es gibt keine "`Grate"' oder schlimmeres.
 \item Reguläre parametrisierte \(p\)-Flächen sind \uline{lokal injektiv}; die Funktionalmatrix 
\[Dx = \left(\frac{\partial (x^1, \dots , x^n)}{\partial (u^1, \ldots , u^p)}\right) = \left(x_1, \ldots , x_p\right)\]
besitzt überall den Höchstrang \(p\) (Satz über implizite Funktionen). Bei lokalen Untersuchungen kann man stets annehmen, dass \(x \colon G \subset \IR^p \rightarrow x \left[G \right] = M \subset \IR^n\) bijektiv ist, also keine Selbstdurchdringungen auftreten. Eine parametrisierte \uline{Hyperfläche}\index{Hyperfläche!regulär} (\(p = n-1\)) im euklidischen \(\IR^n\) ist genau dann regulär, wenn \(\forall_{u} \left(x_1 \times \ldots \times x_{n-1} \right) (u) \neq 0\), d.h. wenn überall der \uline{Normalen(einheits)vektor} 
\[N = \frac{x_1 \times \ldots \times x_{n-1}}{|x_1 \times \ldots \times x_{n-1}|}\] 
existiert.
\end{enumerate}
\end{bemerkung}

\begin{bsp}
Die Abbildung 
\[\left( u.v \right) \in \left]-\pi , + \pi \right[ \times \left]-\frac{\pi}{2}, \frac{\pi}{2} \right[ \mapsto x (u, v) = \begin{pmatrix}
                                                                                                                \cos u \cos v\\
														  \sin u \cos v\\
														   \sin v
                                                                                                               \end{pmatrix} \in \IR^3
\] 
ist (wegen \(|x_1 \times x_2|(u,v) = \ldots = \cos v > 0\)) eine reguläre Parametrisierung der Kugelfläche (\uline{2-Sphäre}\index{Sphäre!2-} \(S^2 \subset \IR^3\)), die aber einen Meridian (samt Polen) auslässt.(\(u =\) geographische Länge, \(v=\) geographische Breite, Würzburg: \(u \approx 10^\circ, v \approx 50^\circ\))\\
Es gibt keine globale, injektive Parametrisierung der \(S^2\). Als ganzes ist sie eine \underline{"`differenzierbare} \underline{Mannigfaltigkeit"'}, die sich nur lokal so parametrisieren lässt.
\end{bsp}

\begin{definition}[\(\C^r\)-Äquivalenz]\index{Fläche!Äquivalenz}
\(\C^r\)-Äquivalenz zweier \(\C^r\)-\(p\)-Flächen
\(\begin{cases}
x \colon G \rightarrow \IR^n, u \mapsto x(u)\\
\widetilde x \colon \widetilde G \rightarrow \IR^n, \widetilde u \mapsto \widetilde x( \widetilde u)
\end{cases}\). \\
Es existiert ein orientierungstreuer \(\C^r\)-Diffeomorphismus 
\[\Phi \colon  G \rightarrow \widetilde G, u \mapsto \widetilde u (u) = \Phi (u)\] 
mit \(x = \widetilde x \circ \Phi\), d.h. 
\[\forall_{n} \, x(u) = \widetilde x (\Phi(u)) = \widetilde x (\widetilde u)\]
\end{definition}

\begin{bemerkung} \(\)
\begin{itemize}
 \item Für \(r \geq 1\) bestimmt die Funktionalmatrix 
 \[D\Phi = \left( \frac{\partial \widetilde u^\sigma}{\partial u^\varrho}\right)_{\sigma, \varrho = 1, \dots , p}\] 
 den Übergang zwischen den Tangentialvektoren \(x_1, \dots , x_p\) und \(\widetilde x_1, \dots , \widetilde x_p\) bezüglich verschiedener Parametrisierungen. Nach der Kettenregel gilt
\[
 x_{\varrho}(u) = \sum_{\sigma = 1}^{p} \widetilde x_\sigma \big(\Phi(u)\big) \frac{\partial \widetilde u^\sigma}{\partial u^\varrho} (u) = \uline{\sum_{\sigma = 1}^{p} \widetilde u_\varrho^\sigma (u) \widetilde x_\sigma \big(\Phi(u)\big)}
\]
(\uline{Basistransformationsformel}\index{Tangentialraum!Basistransformation} im Tangentialraum)
 \item \(\Phi\) orientierungstreu \(\Leftrightarrow \det D\Phi = \det \left(\widetilde u_\varrho^\sigma\right) > 0\)
\end{itemize}
\end{bemerkung}

\begin{definition}
Eine (orientierte, reguläre) \uline{\(\C^r\)-\(p\)-Fläche} im affinen \( \IR^n\) ist eine Äquivalenzklasse regulärer, parametrisierter \(\C^r\)-\(p\)-Flächen \(x \colon G \subset \IR^p \rightarrow \IR^n\).
\end{definition}

\uline{Bedauerlich:}
In der Flächentheorie gibt es \uline{keine ausgezeichnete Parametrisierung} vgl. der Bogenlängenparametrisierung in der Kurventheorie. \\
Deswegen: möglichst \uline{parameterunabhängige} Formulierung von Eigenschaften/Größen. 

\begin{bsp}
 Die \uline{Tangentialräume} (für \(p = 2\) die Tangentialebenen)
 \[
  T_u x := \left[\xcancel{x(u)} + \right] \hl x_1(u) , \dots, x_p(u) \hr
 \]
 sind von der Parametrisierung unabhängig,
 \[
  T_u x = T_{\widetilde u}\widetilde x = \hl \widetilde x_1( \widetilde u) , \dots, \widetilde x_p ( \widetilde u) \hr
 \]
 nicht aber die Basisvektoren \(x_1, \dots, x_p\). \\
 \uline{Tangentialvektoren}
 \[
  X \in T_u x = T_{\widetilde u} \widetilde x \quad \text{(invariante Objekte)}
 \]
 besitzen dann verschiedene Basisdarstellungen:
 \begin{align*}
  X = \sum_{\varrho = 1}^p X^\varrho x_\varrho (u) = \sum_{\sigma = 1}^p \widetilde X^\sigma \widetilde x_\sigma ( \widetilde u)
 \end{align*}
 und aus der Basistransformation
 \begin{align*}
  x_\varrho = \sum_\varrho \widetilde u_\varrho^\sigma \widetilde x_\sigma
 \end{align*}
 folgt die \uline{Koordinatentransformation}
 \[
  \widetilde X^\sigma = \sum \widetilde u_\varrho^\sigma X^\varrho
 \]
 für Vektoren.
\end{bsp}

\begin{bemerkung}
 Zur Vereinfachung der Schreibweise benutzt man beim Rechnen in Koordinaten ("`Tensorkalkül"') meist (wir nicht immer, aber immer öfter) die sogenannte \textsc{Einstein}sche Summationskonvention.\index{Einsteinsche Summationskonvention} \\
 In einem Term wird über Indizes, die einmal "`oben"', einmal "`unten"' vorkommen generell summiert.
 \begin{bsp}
  \begin{align*}
  &\cancel{\sum} X^\varrho \widetilde x_\varrho, \quad \cancel{\sum_\mu} a^{\varrho \mu} b_{\mu \sigma}, \quad \cancel{\sum} b^{\varrho}_{\, \varrho} \text{ Spur} \\
  & \cancel{\sum_{\varrho, \sigma}} X^\varrho \gamma_{\varrho\,\sigma}^{\,\mu} Y^\sigma, \cancel{\sum_\mu} g_{\mu \nu} R_{\varrho \, \sigma \tau}^{\,\mu} = R_{\varrho \nu \sigma \tau}
  \end{align*}
 \end{bsp}
\end{bemerkung}

\subsection{Flächeninterne Metrik auf $p$-Flächen im euklidischen $\IR^n$}
Die euklidische Struktur im \(\IR^n\) induziert in den \(p\)-dimensionalen Tangentialräumen \(T_u x \subset: \IR^n\) punktale Skalarprodukte.

\begin{satz}\label{satz211}
 In jedem Punkt \(x(u)\) einer \(\C^1\)-\(p\)-Fläche im euklidischen \(\IR^n\) wird durch
 \begin{align*}
  I_u \colon T_u x \times T_u x \to \IR, \, (X,Y) \mapsto I_u (X,Y) := \langle X, Y \rangle
 \end{align*}
ein (von der Parametrisierung unabhängiges) Skalarprodukt\index{Fläche!Skalarprodukt} definiert. \\
Das Feld \(u \mapsto I_u\) dieser Bilinearformen heißen \uline{1. Grundform}\index{1. Grundform} (\uline{1. Fundamentalform}) der \(p\)-Fläche. Die (parameterabhängigen) Koordinatenfunktionen
\[
 u \mapsto \uline{g_{\varrho \sigma} (u)} := I_u ( x_\varrho, x_\sigma) = \uline{\langle x_\varrho, x_\sigma \rangle}
\]
bilden überall eine symmetrische, positiv definite Matrix \(\big(g_{\varrho \sigma}(u)\big)\) mit Determinante
\[
 g(u) := \det \big(g_{\varrho \sigma}(u)\big) = a_p^2 (x_1, \dots, x_p)(u) > 0
\]
und für Tangentialvektoren
\[
 X = \sum X^\varrho x_\varrho (u), \, Y = \sum Y^\sigma x_\sigma(u)
\]
erhält man die Koordinantendarstellung
\begin{align*}
I_u (X,Y) = \sum g_{\varrho \sigma}(u) X^\varrho Y^\sigma
\end{align*}
\end{satz}

\begin{beweis}[von Satz \ref{satz211}]
 \(I_u\) ist die Einschränkung des Skalarproduktes im \(\IR^n\) au den Teilraum \(T_u x \subset: \IR^n\) und \(\big(\grs(u)\big)\) ist die Darstellungsmatrix von \(I_u\) bezüglich der speziellen Basis \((x_1, \dots, x_p)(u)\). Weiter gilt
 \[
  I( X,Y) = \left\langle\sum X^\varrho x_\varrho, \sum Y^\sigma x_\sigma \right\rangle = \sum X^\varrho Y^\sigma \underbrace{\langle x_\varrho, x_\sigma\rangle}_{\grs}
 \]
\end{beweis}
 Früher oft übliche Bezeichnung bei \(p = 2\):
 \begin{align*}
  (\grs) = \begin{pmatrix}
            E & F \\
            F & G
           \end{pmatrix}
 \end{align*}
\subsubsection{Mit Hilfe der 1. Grundform lassen sich \uline{flächenintern} berechnen}
\uline{\(\rightarrow\) bei tangentialen Objekten}
\begin{itemize}
 \item die Länge eines Tangentialvektors \mat{X = \sum X^\varrho x_\varrho (u) \in T_u x}
 \[
  |X| = \sqrt{\sum \grs(u) X^\varrho Y^\sigma} \quad (\text{analog der Winkel zwischen Tangentialvektoren }X,Y \ne 0)
 \]
 \item den \(p\)-dimensionalen \uline{Flächeninhalt} eines von Vektoren \(X_1, \dots, X_p \in T_u x\) aufgespannten \(p\)-dimensionalen Parallelogramms
 \[
  \uline{a_p ( X_1, \dots, X_p)} = \sqrt{\det \left( \left\langle X_\varrho, X_ \sigma \right\rangle \right)} = \uline{\sqrt{g(u)} \det \left(X_\beta^\alpha\right)}
 \]
 denn
 \begin{align*}
  &X_\varrho = \sum X_\varrho^\mu x_\mu(u) \Rightarrow \langle X_\varrho, X_\sigma \rangle = \sum g_{\mu \nu} (u) X_\varrho^\mu X_\sigma^\nu  \\
  &\Rightarrow \det \left(\langle X_\varrho, X_\sigma \rangle \right) = \det \big(g_{\mu \nu}(u)\big) = {\det}^2 \left(X_\varrho^\mu \right)
 \end{align*}
 Die explizite Kenntnis der räumlichen Basisvektoren \(x_\varrho (u) \in \IR^n\) ist jeweils unnötig.
\end{itemize}
\uline{\(\rightarrow\) bei Objekten innerhalb der Fläche}
\begin{itemize}
 \item die Länge eines Kurvenstücks
 \[
  L_a^b (c) = \int_a^b \sqrt{\sum \grs\big(u(t)\big) \dot u^\varrho(t) \dot u^\sigma(t)} \dd t
 \]
 Wegen \(c = x \circ u\) gilt
 \begin{align*}
  &\dot c(t) = \sum_\varrho x_\varrho \big(u(t)\big) \dot u^\varrho (t) \\
  \Rightarrow | &\dot c (t) |^2 = \sum \grs \big(u(t)\big) \dot u^\varrho (t) \dot u^\sigma (t)
 \end{align*}
(Ähnlich der Winkel zwischen Flächenkurven)
 \item der \(p\)-dimensionale \uline{Flächeninhalt eines Bereiches} \(x[K] \subset M\)
 \[
  \uline{A_p\big(x[K]\big)} = \int\limits_K a_p \big(x_1(u), \dots, x_p(u) \big) \dd u^1 \dots \dd u^p = \uline{\int\limits_K \sqrt{g(u)} \dd u}
 \]
 Man braucht die Kurve bzw. den Bereich nur in der Parameterebene zu kennen.
\end{itemize}
\textbf{\uline{Ergebnis}:} Lassen sich zwei (im Raum völlig verschiedene) Flächen so auf gleiche Parameter beziehen, dass sie überall die gleiche 1. Grundform-Matrix besitzt, so erhält man gleiche Ergebnisse beim Messen von Längen, Winkeln, Flächeninhalten usw. \\
Durch solche Messungen innerhalb der Fläche kann mann nicht entscheiden, auf welcher Fläche man sich wirklich befindet.

\begin{definition}\(\)
\begin{enumerate}
 \item[a)] Zwei \(\C^1\)-\(p\)-Flächen heißen \uline{isometrisch}\index{Isometrie}, wenn Parametrisierungen
 \[
  u \in G \mapsto x(u) \in M \quad \text{und} \quad u \in G \mapsto \overline x(u) \in \overline M
 \]
 existieren mit 
 \[
  \forall_n \big(\grs (u) \big) = \big(\overline g_{\varrho \sigma} (u) \big)
 \]
 \item[b)] Zwei isometrische \(\C1\)-\(p\)-Flächen mit Parametrisierungen
 \[
  u \in G \mapsto x(u) \in M \quad \text{und}\quad u \in G \mapsto \overline x (u) \in \overline M
 \]
 heißen \uline{ineinander verbiegbar}\index{verbiegbar} (abwickelbar), wenn eine stetige Schar
 \[
  \alpha \in [a,b] \mapsto (\,^\alpha x \colon G \to M_\alpha)
 \]
 isometrischer \(\C^1\)-\(p\)-Flächen existiert mit
 \[
  \,^a x = x, \,^b x = \overline x
 \]

\end{enumerate}

\end{definition}
